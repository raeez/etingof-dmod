\documentclass[etingof-dmod.tex]{subfiles}
\begin{document}
\section{The functors $\pi^!$,$\pi_*$, $\Du$ and $\boxtimes$}
\begin{thm}
  The functors $\pi^!$,$\pi_*$, $\Du$ and $\boxtimes$ preserve the derived
  category of holonomic $\mc{D}$-modules.
\end{thm}
\begin{proof}
  The theorem is clear for $\boxtimes$. For $\Du$ this is a local statement, so
  it reduces to the affine case where we already know it. For $\pi: X
  \rightarrow Y$, the case of $\pi^!$ reduces to the cases when both $X$ and $Y$
  are affine, and again we already know it. It remains to show the property for
  $\pi_*$. Any $\pi$ is a composition of a closed embedding, an open embedding
  and a projection $$\P^N\times Y \rightarrow Y.$$ The case of a closed
  embedding is local, so again reduces to the affine case. In the case of an open
  embedding $$j : U \rightinclusion X$$ we may assume that $X$ is affine. Our
  strategy will be to again reduce to the affine case; cover $U$ by affine open
  sets$$U = \bigcup_{\alpha=1}^n U_{\alpha},$$ and consider the Cech complex
  $$C_k = \bigoplus_{\{\alpha_1,
  \ldots, \alpha_k\}} j_{\alpha_1,\ldots,\alpha_k *}
  \restr{M}{U_{\alpha_1}\cap\cdots\cap U_{\alpha_k}}$$ then by definition $C$
    represents $j_*(M)$. We know that $C_k$ are holonomic, so $C$ is holonomic,
    i.e.\ $j_* M$ is holonomic. Now in the case of the projection $$\pi: \P^N
    \times Y \rightarrow Y$$, we may assume that $Y$ is affine, and we already
    know the statement for $\pi' : \A^N \times Y \rightarrow Y$, so we proceed
    by induction. Let $M$ on $\P^N \times Y$ be holonomic, and $j: \A^N \times Y
    \rightinclusion \P^N \times Y$, then we have an exact triangle $$M
    \rightarrow j_*M \rightarrow N$$ where $N$ is supported on $\P^{N-1}\times
    Y$, so we have $$\pi_* M \rightarrow \pi_* j_* M \rightarrow \pi_*
    N.$$ Now $\pi_*N$ is holonomic by our induction hypothesis, and $\pi_* j_* M
    = (\pi \circ j)_* M$ is holonomic by reduction to the affine case, so $\pi_*
    M$ is holonomic, as desired.
\end{proof}

\begin{rmk}
  Let $\pi: X \rightarrow \pt$ where $X$ is smooth of dimension $n$, then
  $$H^i(\pi_*\O) = H^{i+n}(X,\C)$$ so by poincar\'e duality $$H^i(\pi_!\O) =
  H^{-i}(\pi_*\O)^* = H^{n-i}(X,\C)^* = H_c^{n+i}(X,\C).$$ Therefore, $\pi_!$ is
  called the direct image with compact support, and its right adjoint $\pi^!$ is
  called inverse image with compact support. If instead we work on a singular
  variety, instead of $\O$ we can take $\IC_X$, then we will get the
  intersection cohomology $\IH^*$ (resp\. the intersection cohomology with
  compact support $\IH_c^*$) with the appropriate shift.
\end{rmk}

\subsection{An example}
  Let $X \rightinclusion \A^2$ be given by $xy = 0$, then $X = X_1 \cup X_2$
  with $X_1 = \{x=0\}$ and $X_2 = \{y=0\}$. We have $$\IC_X = i_{1*}\O \oplus
  i_{2*}\O$$ for $$i_j : X_j \rightinclusion X.$$ Note that this scenario
  satisfies the required conditions. So $$\dim \IH^i(X) = \begin{cases} 2 &
    \text{ if } i =
    0 \\
  0 & \text{otherwise} \end{cases}$$
  So we see this cohomology theory is different from the ordinary cohomology.
  \subsection{another example}
 Let $\Gamma$ be a finite group and let $X$ be a smooth $\Gamma$-variety
  such that $\Gamma$ acts on $X$ freely at the generic point. Assume that $Y = X /
  \Gamma$ is a variety, then $\Gamma$ acts on $\pi_* \O_X$. Let $V
  \rightinclusion Y$ be the open set of points coming from the locus of trivial
  stabilizers
  \begin{lem}
    $$j_{!*} ( \restr{\pi_* \O_X}{V}) = \pi_* \O_X$$
  \end{lem}
  \begin{proof}
    Let $\hat{j} : U \rightinclusion X$ where $U = \pi^{-1}(V)$. Since $\pi$ is a
    finite morphism, it is proper, hence $$j_! ( \restr{\pi_* \O_X}{V}) = \pi_*
    ( \hat{j}_!(\restr{\O_X}{U})$$
      $$j_*( \restr{\pi_* \O_X}{V}) = \pi_*( \hat{j}_*
      \restr{\O_X}{U})$$
      so $$\Im(j_! (\pi_* \restr{\O_X}{V}) \rightarrow j_*(\pi_* \restr{O_X}{V})
        = \pi_*(\Im(\hat{j}_! \restr{\O_X}{U} \rightarrow \hat{j}_*
        \restr{\O_X}{U})) = \pi_*\O_X$$
        since $X$ is smooth. This means that $$j_{!*}( \restr{\pi_*\O_X}{V}) =
        \pi_*\O_X$$
  \end{proof}

  \begin{cor}
    $$\IC_Y = (\pi_*\O_X)^{\Gamma}$$
  \end{cor}
  \begin{cor}
    $$H^i(\IC_Y) = \IH^{i}(Y)$$ so $$H^{\dim X+i}(Y,\C) =
    H^{\dim X+i}(X,\C)^{\Gamma}$$
  \end{cor}

  \subsection{another example}
    Let $Y \rightinclusion \C^3$ be defined by $xy - z^2 = 0$, then $$Y = \C^2 /
    {\Z/2\Z}$$ and so $$\IH^*(Y) = H^*(Y)$$ i.e.\ coincides with the usual
    cohomology.

  \subsection{another example}
    Let $\pi: X \rightarrow Y$ be a morphism of irreducible algebraic varieties.
    We say that $\pi$ is small if $$\codim \{y \in Y | \dim \pi^{-1}(y) \geq m\}
    \geq 2m+1$$

    \begin{prop}
      Suppose $\pi : X \rightarrow Y$ is a small resolution of singularities,
      then $$\pi_* \O_X  = \IC_Y$$ so $$\IH^*(Y) = H^*(X)$$
    \end{prop}

    \begin{ex}
      Let $\lie{g}$ be a simple lie algebra, and $$\widetilde{\lie{g}} =
      \{(x,\lie{b})
      | x \in \lie{g}, x \in \lie{b}, \lie{b} \text{ a borel
    subalgebra}\}$$ then we have a map $$\pi: \widetilde{\lie{g}} \rightarrow
    \lie{g} \times_{\lie{h}/W} \lie{h}$$ given by $\pi(x,\lie{b}) = (x,\hat{x})$
    where $\hat{x}$ is the image of $x$ under the canonical quotient
    $$\lie{b}/[\lie{b},\lie{b}] = \lie{h}$$ i.e.\ the standard cartan. It is
    known that $\pi$ is a small resolution called the \textit{Grothendieck
    simultaneous resolution}, thus $$\IH^*(\lie{g}\times_{\lie{h}/W}\lie{h}) =
    H^*(\widetilde{\lie{g}}) = H^*(G/B)$$
    \end{ex}

    \begin{ex}
      $\lie{g} = \lie{sl}_2$, then $\lie{g}\times_{\lie{h}/W}\lie{h}$ is a
      double cover of $\lie{g}$ branched over the nilpotent cone $xy + z^2 = 0$,
      hence it is the surface $t^2 = xy + z^2$---a homogenous quadric $Q$ in
      $\C^4$. Also, note that $\widetilde{g}$ in this case is the total space of
      $\O(-1)\oplus\O(-1)$ as a bundle over $\P^1$. It follows that $$\IH^j(Q) =
      \begin{cases} \C & \text{ if } j = 0 \text{ or } j=2\\
      0 & \text{otherwise}\end{cases}$$ while $$H^j(Q) = \begin{cases} \C
        &\text{ if } j = 0\\
      0 & \text{otherwise}\end{cases}$$
    \end{ex}

\end{document}
